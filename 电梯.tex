\documentclass{article}   % 请用 xelatex 命令编译
\setlength{\parskip}{0.5em} 
\usepackage[UTF8]{ctex}
\usepackage{color}

\usepackage{tcolorbox}
\tcbuselibrary{skins,breakable,theorems}

\usepackage{listings}
\usepackage{geometry}
\geometry{a4paper,left=2.8cm,right=2.6cm,top=3.7cm,bottom=3.5cm}

\begin{document}


26栋三单元安装电梯的过程是个很好的中篇小说。可惜我驾驭不了。
	
各种奇葩人。面和心不和。人性。而且可以写成排列组合。

话说疫情,四月,到郊区玩,有人说很像十日谈,我们也仿照一下。

环顾周围,国家做了不少好事,这也就不远。比如这个公园,很好。
开车,很好。

比如为了老年化社会,老式福利楼,没电梯,住了不少老人,国家补贴安装电梯,多好。

话说,有个单元,怎么也谈不拢。有人不同意安装。不是不参加,是不同意,说挡着他的采光了。

我很理解他们。

因为当初他们可以选择七楼,之所以选一楼,是七楼难爬。

二零零几年的房子,住了将近二十年了。

一楼是方便,但是潮湿,蚊虫也比楼上多。

家具陆续坏脚,楼前后树长大了,采光也是问题。

年纪大了,住楼上,有电梯,比一楼好多了。

房子的销售价值也不同。一楼很难增值,七楼按了电梯,增值是一定的。

所以,不同意。

好在国家发觉了这个问题,就是,人,不都是好人。有些人是不管别人死活的。
所以不要求全部同意也可以安装。

第四版
\end{document}
