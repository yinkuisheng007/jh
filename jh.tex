\documentclass[11pt]{article}
\usepackage[UTF8]{ctex}
\usepackage{color}
\usepackage{yhmath}
\usepackage{geometry}
\usepackage{amsmath}
\usepackage{tikz}
\geometry{left=2.5cm,right=2.5cm,top=3cm,bottom=3cm}
\usepackage[raggedright]{titlesec}%标题居左
%\usepackage{unicode-math}

\def\newline{\vbox{}}
\setlength{\parskip}{0.2em} %段间距
\linespread{1.5}
%\usepackage{indentfirst} 
\usepackage{multicol} 
%\newcounter{counterA}

%\def\BBB{\heiti\songti{\addtocounter{counterA}{1}\arabic{counterA}}.}

%引用自定义宏包,实现内容和形式的分离
\usepackage{jh}
%\usepackage{unicode-math}
%\setmathfont{STIX Math}
%\newcommand\parallelogram{\mathord{\}} $

%类似于html的超链接
\usepackage[colorlinks,linkcolor=red,anchorcolor=blue,citecolor=green]{hyperref}

\begin{document}


\centerline{\Large 高三平面几何复习题}
\newline

\iffalse
我有一份高考复习资料,草纸、手刻油印,我保存了很多年。

退休了,没事干,想点往事。

想起了李长银,我的初一数学老师,教我几何。也是我大哥的数学老师。
听说他去世得早。留下好几个孩子。当年的孩子早已长大成人,不知道过的怎么样。

王永道,我的初一同学。我们一起在路旁,用树枝,在泥土路面上,探讨几何题。
他上的本地高中。没考上大学。

蒋伦生,我的初二数学老师。风度翩翩。后来在乡上工作。

我的高中数学老师,女的,名字都忘了。形象还记得。她是南京下放户。

三个数学老师都很好。我对他们有美好记忆。

是老师改变我的一生。我这么说是因为初中时的同学王永道,
起码初中时,不比我差。后来差距就大了。也许再后来又比我强。没联系。社会这所大学也是很难说的。

还有很多老师也很好。我只是特别记得这几个老师。和数学有关系。
也有不好的老师,就不说了。


这是一份草纸刻板油印件。高考复习用的。很可能是王月平同学手刻。他只有一只左手。是天生的。
听说他不在了。

77、78年,那时白纸少。据说高考试卷用纸是印刷毛泽东选集的纸。

想起高中化学老师。是他帮我们复习的几何。听说他是上海人,研究生。
没少说他坏话。年少的我,屁都不懂啊。
有一天他拿着三角板走进课堂。


有人说我是``数学家'',或者是``搞数学的''。
这不是什么坏话。
但是我不是。我知道这叫``拔高''。
虽然不是什么坏话,但是不符合实际。
不符合实际,不好。我知道。我信了的话会出问题。
我喜欢实事求是,实事求是,我的理解,也叫名实相符。

我大学学的工程力学。学的不好。大学前两年学习还是认真的,后两年看小说等杂书去了。

工作后不敢用力学解决问题。所以搞搞计算机。
计算机这玩意不兑现。

我的计算机知识是自学的。在别人看来的``学习'',我这里都是``研究''。因为没人指导啊。

大概快退休了,也算入门了。入门后也就靠边了、被凉起来了。

其实,有些人,自称他称专家,其实没入门啊。

我不是计算机专家。

我知道专家长什么样子。

我很想当个专业程序员,或者达到专业程序员的水平。
我是业余的。我不会git。想学,正在学。

自学过程中,接触到unix,linux,emacs,lisp,lex yacc,TeX,tikz,metapost这些。虎书、龙书、鲸书。
很感兴趣。打开一片新天地。

我大概明白图灵机原理,明白冯诺依曼架构是怎么回事。
明白这些和我使用的电脑的关系。但不深入。

中学学的平面几何,至今还有一点记忆。
微积分反而忘了。


所以两者一结合,用快退休时学习的工具,回忆回忆中学知识。
看看能不能用这么好的工具,把几何题表达清楚。

主要是防痴呆。

不要拔高。不成体系,也没有一题多解。说清楚就不错了。

一个字,玩。

我有点困惑。

证明到哪一步才能说清楚?
初中生的证明和高中生的证明有什么不同?
数学家的证明又有什么不同?
需要刨根问底不?
我拿捏不好。

公理、定理?

\fi

\input jh001

\input jh002
\input jh003
\input jh004

\input jh005
\input jh006
\input jh007
\input jh008

\input jh009
\input jh010
\input jh011

\input jh012
\input jh013
\input jh014

\input jh015
\input jh016
\input jh017

\input jh018
\input jh019
\input jh020
\input jh021
\input jh022
\input jh023

\input jh024
\input jh025
\input jh026
\input jh027
\input jh028
\input jh029
 
\input jh030
\input jh031
\input jh032
\input jh033
\input jh034

\input jh035
\input jh036
\input jh037
\input jh038
\input jh039
\input jh040

\input jh041
\input jh042
\input jh043
\input jh044
\input jh045
\input jh046
\input jh047

\input jh048
\input jh049
\input jh050
\input jh051
\input jh052
\input jh053

\input jh054
\input jh055
\input jh056
\input jh057
\input jh058
\input jh059
\input jh060
\input jh061
\input jh062
\input jh063

\input jh064
\input jh065
\input jh066
\input jh067
\input jh068
\input jh069
\input jh070
\input jh071
\input jh072

\input jh073
\input jh074
\input jh075
\input jh076
\input jh077
\input jh078
\input jh079
\input jh080
\input jh081
\input jh082
\input jh083
\input jh084
\input jh085
\input jh086
\input jh087
\input jh088
\input jh089
\input jh090
\input jh091
\input jh092
\input jh093
\input jh094
\input jh095
\input jh096
\input jh097
\input jh098
\input jh099
\input jh100
\input jh101
\input jh102
\input jh103
\input jh104
\input jh105
\input jh106
\input jh107
\input jh108
\input jh109


\input 等腰三角形



%惯用法
\iffalse

$\wideparen{AB}$弧AB
$\angle$
$\triangle$
$\cong $ S=全等
$\parallel$平行
$\perp$垂直
$\simeq$S-
$\sim$S相似
$\frac{分子}{分母}

平行四边形只能用tikz画一个。
定义一个新命令\parallelogram

面积\area

%定义点
\tkzDefPoint(0,0){B} 
\tkzDrawPoints(A,B,C) 
\tkzDrawSegments(A,B B,C C,A)
\tkzDrawLine[option](P1,P2)
\tkzLabelPoints(A)
above:上方
%   below:下方
%   left:左方
%   right:右方
above right 右上

%BA之间的点,离B 0.3
\tkzDefPointBy[homothety=center B ratio 0.3](A)
\tkzGetPoint{D}
%
\tkzDrawPolygon(A,B,C)
%定义园 
\tkzDefCircle[in](A,B,C)
\tkzGetPoint{I} 
\tkzGetLength{ls}
\tkzDrawCircle[R,red](I, \ls pt)

\tkzDrawCircle[option](A,B,C)
% 作关于A、B、C三点的圆。在option中,除了颜色,可选的其他常用选项有:
%   circum:作过A、B、C三点的圆
%   in:作三角形ABC的内切圆

\tkzDrawCircle[options,with nodes](A,P,Q)
% 作以A为圆心,PQ为半径的圆,相当于用圆规量取后画图

%DE和BC之间交点
\tkzInterLL(D,E)(B,C)
\tkzGetPoint{F}
%过D作AC的平行线
\tkzDefPointBy[translation= from A to C](D)	
\tkzGetPoint{H}
%求O到MN的投影,得到A点
\tkzDefPointBy[projection=onto M--N](O)
\tkzGetPoint{A}
%将N点绕A旋转60度
\tkzDefPointBy[rotation=center A angle 60](N)
\tkzGetPoint{E1}
%求园o上C点的切线
\tkzDefTangent[at=C](O)
\tkzGetPoint{C1}
%E 到圆OQ的切线
\tkzDefTangent[from=E](O,Q)
\tkzGetPoints{F}{F1}
%
\tkzDefMidPoint(A,B) 
\tkzGetPoint{E}
%
\tkzDefSpcTriangle[intouch](A,B,C){D,E,F}
\tkzDrawPoints[blue](D,E,F)
%
\tkzMarkAngles[orange,mark=none,size=0.5cm](B,P,D D,P,C)
\tkzLabelAngle(E,P,C){1}
\tkzLabelAngle(B,A,C){$36^\circ$}


%
\tkzAutoLabelPoints[blue,center = I](D,E,F)

\tkzGetFirstPoint{A}
% 获取同时生成的两个点中的第一个点,保存为A
\tkzGetSecondPoint{B}
% 获取同时生成的两个点中的第二个点,保存为B	
%
%交点
\tkzInterLL(B,C)(A,K) 
\tkzGetPoint{F}

\tkzInterCC(A,B)(C,D)
%求A--E1和圆的交点B,C
\tkzInterLC(A,E1)(O,D) 
\tkzGetPoints{B}{C}

角平分线
\tkzDefLine[bisector,normed](B,A,C)
\tkzGetPoint{K}

%过P点作AD的垂线
\tkzDefLine[perpendicular=through P](A,D)
\tkzGetPoint{N}

%标记直角
\tkzMarkRightAngle[blue](D,P,F)


%标记线段
\tkzMarkSegments[blue](D,G C,E D,B)
标记角

\tkzMarkSegments[mark=s||](A,B B,C C,A)
\tkzMarkSegments[color=red,mark=|](A,B C,B)
\tkzMarkSegments[color=blue,mark=||](B,E B,D)
\tkzMarkSegments[color=brown,mark=|||](C,D A,E)


% 标注
\tkzLabelSegments[swap](A,B){$c$}

\tkzDefPoint(<θ:ρ>){A}
\tkzDefShiftPoint[A](<x,y> or <θ:ρ>){B} 或者 
\tkzDefPoint[shift={(<x,y> or <θ:ρ>)}]((<x,y> or <θ:ρ>){B}) 或者 
\tkzDefPointOnCircle[angle=θ,center=A,radius=ρ] 
\tkzGetPoint{B}

带标注的绘制
\tkzDrawSegment[dim={$d$,16pt,above=6pt}](O,P)
\fi		
\end{document}


