
%\columnseprule=1pt         % 实现插入分隔线

1、在$\Delta$ ABC中,AB=AC,在AB上取D点,AC延长线上取E点,使BD=CE,连DE交BC于F,求证:DF=FE。

\begin{twocol}
{

	\begin{tikzpicture}
	\tkzDefPoint(0,0){B} 
	\tkzDefPoint(4,0){C}
	\tkzDefPoint(2,5){A}
	
	\tkzDrawPoints(A,B,C) 
	\tkzDrawSegments(A,B B,C C,A)
	\ShowPoint[above](A)
	
	\tkzLabelPoints[left](B)
	\tkzLabelPoints(C)
	%BA之间的点,离B 0.3
	\tkzDefPointBy[homothety=center B ratio 0.3](A)
	\tkzGetPoint{D}
	%AC之间的点,离C 1.3
	\tkzDefPointBy[homothety=center A ratio 1.3](C)
	\tkzGetPoint{E}
	
	\tkzDrawPoints(D) 
	\tkzLabelPoints[left](D)%above/below + left/right
	
	\tkzDrawPoints(E) 
	\tkzLabelPoints(E)
	
	\tkzDrawSegments(D,E)
	\tkzDrawSegments(C,E)

	%DE和BC之间交点
	%\tkzInterLL(D,E)(B,C)
	%\tkzGetPoint{F}
	%\lljd{D}{E}{B}{C}{F}
	\jiaodian(D,E)(B,C)(F)
	
	\tkzDrawPoints(F) 
	\tkzLabelPoints[above](F)
	%作辅助线
	\tkzDefPointBy[translation= from A to C](D)	%过D作AC的平行线
	\tkzGetPoint{H}
	\tkzInterLL(D,H)(B,C)%DH和BC交点
	\tkzGetPoint{G}
	\tkzDrawSegments[red,dashed](D,G)
	\tkzDrawPoints(G) 
	\tkzLabelPoints[red](G)	
	
	\tkzMarkAngles[orange,mark=none,size=0.5cm](G,B,D D,G,B A,C,B)
	\tkzMarkAngles[red,mark=none,size=0.5cm](G,D,F C,E,F)
	\tkzMarkAngles[green,mark=none,size=0.3cm](F,G,D F,C,E)
	
	\tkzMarkSegments[blue](D,G C,E D,B)
	
\end{tikzpicture}
}
{
	
	
作辅助线 DG平行AE,交BC于G。

因为DG平行AE,

所以$\angle$DGB=$\angle$ACB。\WHY{同位角相等}\hfill(1)

因为$\triangle$ABC为等腰三角形

所以$\angle$ABC=$\angle$ACB\hfill(2)

所以由(1)(2)得出

$\angle$ABC=$\angle$DGB\WHY{等量相等}

所以$\triangle$BDG也是等腰三角形。

所以BD=DG。\WHY{等腰三角形两腰相等}\hfill (3)

根据条件:BD=CE和(3)得出:DG=CE。\hfill (4)

由DG平行AE还可得出:

$\angle$GDF=$\angle$FEC,
$\angle$DGF=$\angle$FCE\WHY{平行线内错角相等}\hfill(5)


由(4)(5)可以得出

$\triangle$DGF$\cong$$\triangle$FCE\WHY{角边角}

所以DF=FE。\WHY{全等三角形对应边相等}

}


\end{twocol}
